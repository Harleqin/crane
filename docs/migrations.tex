\section[title=Migrations]{

Your schema will change. Most ORMs hope the users will be happy running manual
\c{ALTER TABLE} statements or provide migration functionality through an
external plugin (Alembic for SQLAlchemy, South for the Django ORM).

Migrations are completely built into Crane, and are designed to be intrusive:
You redefine the schema, reload, and Crane takes care of everything. If your
migration plan is too complicated for Crane, then you write a simple function
that does some transformations and Crane puts that in its migration history, all
that without ever having to leave your Lisp environment or accessing the shell.

\section[title=Examples]{

\code[lang=lisp]{(deftable employees
  (name :type string :null nil)
  (age  :type integer)
  (address :type string :null nil))
}

Now, if you decide that addresses can be nullable, you just redefine the class
(Make the change, and either `C-c C-c` on Emacs or Quickload your project):

\code[lang=lisp]{(deftable employees
  (name :type string :null nil)
  (age  :type integer)
  (address :type string))
}

And Crane will spot the difference and perform the migration automatically.

}

}
