\section[title=Making Queries]{

\section[title=High-Level API]{

\cldoc{crane.interface:filter}

\cldoc{crane.interface:single}

\cldoc{crane.interface:get-or-create}

}

\section[title=Functional SQL]{

Crane exports the important bits of SxQL so you can write queries using this DSL
without worrying about packages. The syntax is fairly straightforward, and has
few surprises, so a lot of the time consulting the documentation is not
required. It's simply SQL with Lisp syntax.

Examples:

\code[lang=lisp]{
cl-user> (query (select :tonnage
                  (from :ship)
                  (where (:and (:> :tonnage 125)
                               (:<= :tonnage 500)))
                  (order-by :tonnage)
                  (limit 10)))
;; => ((:|tonnage| 445))
}
}
}
